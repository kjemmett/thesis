%!TEX root = ../thesis.tex
\chapter{Introduction}
\label{ch:introduction}

\epigraph{``I predict a new subject of statistical topology. Rather than count the
number of holes, Betti numbers, etc., one will be more interested in the
distribution of such objects on noncompact manifolds as one goes out
to infinity''}{Isadore Singer}

This thesis contains results of applying methods from topological data analysis to various problems in genomics and evolution.
It primarily details the use of persistent homology as a tool to measure the prevalence and scale of nonvertical evolutionary events, such as reassortments and recombinations.
In so doing, various techniques are developed to extract statistical information from the topological complexes that are constructed.
We rely on results from \cite{Grosz_and_Sidner_1986}.

This thesis contains results of applying methods from topological data analysis to various problems in genomics and evolution.
It primarily details the use of persistent homology as a tool to measure the prevalence and scale of nonvertical evolutionary events, such as reassortments and recombinations.
In so doing, various techniques are developed to extract statistical information from the topological complexes that are constructed.
We rely on results from \cite{Grosz_and_Sidner_1986}.

This thesis contains results of applying methods from topological data analysis to various problems in genomics and evolution.
It primarily details the use of persistent homology as a tool to measure the prevalence and scale of nonvertical evolutionary events, such as reassortments and recombinations.
In so doing, various techniques are developed to extract statistical information from the topological complexes that are constructed.
We rely on results from \cite{Grosz_and_Sidner_1986}.

This thesis contains results of applying methods from topological data analysis to various problems in genomics and evolution.
It primarily details the use of persistent homology as a tool to measure the prevalence and scale of nonvertical evolutionary events, such as reassortments and recombinations.
In so doing, various techniques are developed to extract statistical information from the topological complexes that are constructed.
We rely on results from \cite{Grosz_and_Sidner_1986}.

This thesis contains results of applying methods from topological data analysis to various problems in genomics and evolution.
It primarily details the use of persistent homology as a tool to measure the prevalence and scale of nonvertical evolutionary events, such as reassortments and recombinations.
In so doing, various techniques are developed to extract statistical information from the topological complexes that are constructed.
We rely on results from \cite{Grosz_and_Sidner_1986}.

This thesis contains results of applying methods from topological data analysis to various problems in genomics and evolution.
It primarily details the use of persistent homology as a tool to measure the prevalence and scale of nonvertical evolutionary events, such as reassortments and recombinations.
In so doing, various techniques are developed to extract statistical information from the topological complexes that are constructed.
We rely on results from \cite{Grosz_and_Sidner_1986}.

This thesis contains results of applying methods from topological data analysis to various problems in genomics and evolution.
It primarily details the use of persistent homology as a tool to measure the prevalence and scale of nonvertical evolutionary events, such as reassortments and recombinations.
In so doing, various techniques are developed to extract statistical information from the topological complexes that are constructed.
We rely on results from \cite{Grosz_and_Sidner_1986}.

This thesis contains results of applying methods from topological data analysis to various problems in genomics and evolution.
It primarily details the use of persistent homology as a tool to measure the prevalence and scale of nonvertical evolutionary events, such as reassortments and recombinations.
In so doing, various techniques are developed to extract statistical information from the topological complexes that are constructed.
We rely on results from \cite{Grosz_and_Sidner_1986}.

This thesis contains results of applying methods from topological data analysis to various problems in genomics and evolution.
It primarily details the use of persistent homology as a tool to measure the prevalence and scale of nonvertical evolutionary events, such as reassortments and recombinations.
In so doing, various techniques are developed to extract statistical information from the topological complexes that are constructed.
We rely on results from \cite{Grosz_and_Sidner_1986}.

