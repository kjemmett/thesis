%!TEX root = ../thesis.tex
\chapter{Introduction}
\label{ch:introduction}

\epigraph{``I predict a new subject of statistical topology. Rather than count the
number of holes, Betti numbers, etc., one will be more interested in the
distribution of such objects on noncompact manifolds as one goes out
to infinity''}{Isadore Singer}

This thesis contains results of applying methods from topological data analysis to various problems in genomics and evolution.
It primarily details the use of persistent homology as a tool to measure the prevalence and scale of nonvertical evolutionary events, such as reassortments and recombinations.
In so doing, various techniques are developed to extract statistical information from the topological complexes that are constructed.
We rely on results from \cite{Grosz_and_Sidner_1986}.

The rest of this thesis is organized as follows.
In Chapter \ref{ch:background} we present background information on the wide range of topics discussed in this thesis.
This includes discussion of 

In Part \ref{part:theory}, we discuss two important problems.
In Chapter \ref{ch:parametric_inference} we develop an approach for performing statistical inference using the information contained in the persistence diagram.
In Chapter \ref{ch:complex_construction} we propose two methods of complex construction that generalize the traditional Vietoris-Rips but have favorable properties for phylogenetic analysis.
In Part \ref{part:applications_microorganism} we apply our approaches to several datasets deriv
In this section, we consider two models.
In Part \ref{part:applications_human}, we apply our approaches to a several problems in human population genetics and biology.
In Chapter \ref{ch:human_recombination_rate} we measure the human recombination rate.
In Chapter \ref{ch:human_population_structre} we reconstruct models of human demographic movements.
In Chapter \ref{ch:human_chromatin_folding} we analyze Hi-C data to understand patterns of chromatin folding in the nucleus.
Finally, in Chapter \ref{ch:conclusions} we summarize our results and present possible avenues for future directions.