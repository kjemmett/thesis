%!TEX root = ../thesis.tex

\chapter{Conclusions}
\label{ch:conclusions}

This thesis has considered the problem of characterizing reticulate modes of evolution in large-scale genomics data.
We have drawn on methods from topological data analysis, including persistent homology to quantify scales and frequencies of reticulation, and Mapper, to provide a condensed representation of molecular relationships.
In Part~\ref{part:theory}, we developed several theoretical approaches for analyzing data using TDA. 
In Chapter~\ref{ch:complex_construction} we developed alternative topological complex constructions in order to increase the sensitivity of persistent homology.
In Chapter~\ref{ch:parametric_inference} we developed a framework for statistical inference using the peristence diagram.
We used this to develop an estimator for the recombination rate in the coalescent model, a common stochastic model in population genetics.
In Part~\ref{part:applications}, we applied our general framework to large-scale genomic datasets.
These included microorganisms: bacteriophage, influenza, and pathogenic bacteria.
