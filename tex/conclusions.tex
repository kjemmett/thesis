%!TEX root = ../thesis.tex

\chapter{Conclusions}
\label{ch:conclusions}

This thesis has considered the problem of characterizing reticulate modes of evolution in large-scale genomics data.
We have drawn on methods from topological data analysis, specifically persistent homology to quantify the scale and frequency of reticulate events, and Mapper to provide condensed representations of molecular relationships.
In Part~\ref{part:theory}, we developed several theoretical approaches for analyzing data using TDA. 
In Chapter~\ref{ch:complex_construction} we developed alternative topological complex constructions in order to increase the sensitivity of persistent homology.
In Chapter~\ref{ch:parametric_inference} we developed a framework for statistical inference using the peristence diagram.
We used this to develop an estimator for the recombination rate in the coalescent model, a common stochastic model in population genetics.

In Part \ref{part:applications}, we applied our general approach to several problems in evolution and genomics.
In Chapter \ref{ch:phage} we studied phages, viruses of single-celled microorganisms.
We showed how persistent homology can recover inconsistencies in existing morphology-based taxonomies, used a network approach to construct an alternative genome-based representation of phage relationships, and identified representative gene families conserved within phage populations.
In Chapter \ref{ch:influenza} we studied influenza, a common human pathogen.
We showed how persistent homology can capture widespread patterns of reassortment, including nonrandom cosegregation of segments and barriers to subtype mixing.
In contrast to traditional influenza studies, which have focused on the phylogenetic branching patterns of only the two surface-marker proteins, we used Mapper combined with whole-genome data to represent influenza molecular relationships.
We identified unexpected relationships between divergent influenza subtypes.
In Chapter \ref{ch:pathogens} we studied pathogenic bacteria.
We used two sources of data to measure rates of reticulation in both the core genome and the mobile genome across a range of species.
Mapper is used to represent the population of \emph{S. aureus} and analyze the spread of antibiotic resistance genes.
The potential for the spreading of antibiotic resistance in the human microbiome is investigated.