%!TEX root = ../thesis.tex
\subsection{Mapper}


The field of exploratory data analysis is 
\emph{Mapper} is an algorithm for the representation of
The \emph{Mapper} algorithm belongs to the class of algorithms designed for \emph{condensed representation} and \emph{exploratory data analysis}.

In contrast to persistent homology, which summarizes high-dimensional data 

Mapper allows for qualitative analysis of high-dimensional data through direct visualization.
In this sense it belongs within the larger category of dimensionality reduction techniques such as multidimensional scaling (MDS) and their nonlinear extensions, including Isomap and t-SNE.
However, mapper has certain advantages over these approaches, as we will see.

Taking an explicitly topological approach has three key advantages: coordinate freeness, invariance to deformation, and compressed representation of shape.

(1) Coordinate free
(2) Invariance to deformation - robustness to noise
(3) Compressed representation - ability to handle large datasets.

Compressed representation: if our dataset is large, a network analysis may be difficult to .
Mapper allows us to control the resolution at which we explore the data.

*multiresolution*


In this way, mapper is well suited for interactive analysis and visualization.



The \emph{Mapper} algorithm was developed by Gurjeet Singh and Gunnar Carlsson in \cite{Singh:2007ve}.

High-dimensional data to graph/network representation.


Mapper was first applied to problems in RNA folding in \cite{Bowman:2008esa} and breast cancer subtype identification in \cite{Nicolau:2011}.
One of the earliest applications of mapper can be found in \cite{Nicolau:2011}, wher


In our work we use the commercial implementation of Mapper developed by Ayasdi \cite{AyasdiIris:2015}.
An open-source implementation of Mapper is available in the Python Mapper package \cite{Mullner:2013}.


Mapper: a mathematical tool that builds a simple geometric representation of data along preassigned guiding functions called filters. Mapper provides both a method for mathematical data analysis and a visualization tool; the filter functions introduced through Mapper define a framework for supervised analysis. Approximates a collapse of the data into a simple, low dimensional shape, and the filter functions act as guides along which the collapse is done


Mapper is coordinate free and depends only on the similarity of points as measured by the distance function.
Further exposition can be found in \cite{Lum:2013cz}.



Steps:
(1) Project using filter function.
(2) Create overlapping bins
(3) Cluster in the projected space.
(3) Connect pairs of bins with shared points



How does mapper work:

