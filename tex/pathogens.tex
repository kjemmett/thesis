Pathogenic bacteria can lead to severe infection and mortality and presents an enormous burden on human populations and public health systems.
One of the achievements of twentieth century medicine was the development of a wide range of antibiotic drugs to control and contain the spread of pathogenic bacteria, leading to vastly increased life expectancies and global economic development.
However, rapidly rising levels of multidrug antibiotic resistance in several common pathogens, including \emph{Escherichia coli}, \emph{Klebsiella pneumoniae}, \emph{Staphylococcus aureus}, and \emph{Neisseria gonorrhoea}, is recognized as a pressing global issue with near-term consequences \cite{Neu:1992gk,Thomas:2005hp,WHO:2014wa}.
The threat of a post-antibiotic 21st century is serious, and new methods to characterize and monitor the spread of resistance are urgently needed.

Antibiotic resistance can be acquired through point mutation or through horizontal transfer of resistance genes.
Horizontal exchange occurs when a donor bacteria transmits foreign DNA into a genetically distinct bacteria strain.
Three mechanisms of horizontal transfer are identified, depending on the route by which foreign DNA is acquired \cite{Ochman:2000dr}.
Foreign DNA can be acquired via uptake from an external environment (transformation), via viral-mediated processes (transduction), or via direct cell-to-cell contact between bacterial strains (conjugation).
Resistance genes can be transferred between strains of the same species, or can be acquired from different species in the same environment.
While the former is generally more common, an example of the latter is the phage-mediated acquisition of Shiga toxin in \emph{E. coli} in Germany in 2011 \cite{Rohde:2011ju}.
Elements of the bacterial genome that show evidence of foreign origin are called genomic islands, and are of particular concern when associated with phenotypic effects such as virulence or antibiotic resistance.

The presence of horizontal gene transfer precludes accurate phylogenetic characterization, because different segments of the genome will have different evolutionary histories.
Bacterial species definitions and taxonomic classifications are made on the basis of 16S ribosomal RNA, a highly conserved genomic region between bacteria and archaea species \cite{Woese:1977vd}.
However, the region generally accounts for less than 1\% of the complete genome, implying that the vast majority of evolutionary relationships are not accounted for in the taxonomy \cite{Dagan:2006up}.
Because of the important role played by lateral gene transfer, new ways of characterizing evolutionary and phenotypic relationships between microorganisms are needed.