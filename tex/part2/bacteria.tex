%!TEX root = ../../thesis.tex
\chapter{Prokaryote Reticulate Evolution - Tree of Life}
\label{ch:prokaryotes}

In this chapter we examine evolutionary relationships across the prokaryotic domain.
As input data, we use the Cluster of Orthologous Genes (COG) database at NCBI \cite{Galperin:2014ua}.
Using a combination of topological tools, we present a construction meant to extend the tree of life paradigm.

\section{Introduction}

In this chapter, we examine evolutionary relationships across the prokaryotic domain.

First, we use persistent homology to characterize reticulation.
Second, we use mapper to visualize evolutionary relationships.

\section{Materials and Methods}

As input data, we use the Cluster of Orthologous Genes (COG) database from NCBI \cite{Galperin:2014ua}

\section{Results}

To visualize relationships, we use the Mapper algorithm, as implemented in Ayasdi Iris.

\section{Conclusion}

In this chapter, we have examined evolutionary relationships across the prokaryotic domain.