%!TEX root = ../thesis.tex
Reticulate modes of genetic exchange can confound traditional methods of establishing evolutionary relationships among sets of related organisms.
Increasing data has pointed to the prevelance of these modes, and underscored the lack of a unified mathematical approach to capturing and representing the scale and frequency of these events.
This thesis contains results of applying new mathematical methods drawn from applied and computational topology to the problem of measuring reticulate evolution in molecular sequence data.
In so doing, new techniques are established for constructing topological representations and extracting statistical patterns from biological data sets.
We apply our approaches to several types of molecular sequence data.
We first 
Second, we study reassortment in influenza virus.
We measure rates of reassortment, identify patterns of nonrandom segment cosegregation, and.
Third,


We apply our approaches to several types of molecular sequence data, include bacteriophage, influenza, and pathogenic bacteria.
We also consider patterns of intranuclear chromatin folding in bacteria and humans.